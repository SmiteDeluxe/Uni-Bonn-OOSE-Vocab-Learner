%*******10********20********30********40********50********60********70********80

% For all chapters, use the new defined chap{} instead of chapter{}
% This will make the text at the top-left of the page be the same as the chapter

\chap{User Documentation}
The developed application implements the \href{https://api.voc5.org/}{api.voc5 API} offering a client-side graphical user interface (GUI) to the functionality of creating, managing, and learning user-specific vocabulary lists. 

    \section{Requirements}
    This application is a java program, hence, requires an installation of java (version 13 or higher.) Also internet access is required while using the application. 

    \section{Login}
    The application stores data remotely on a api.voc5 implementing server at the selected address. In order to manage the personal data an user-account needs to be created by inserting a valid email address (needs to contain a '@') and password, and pressing "Register". \textbf{User data is not stored safely - Make sure to not use real data!} \\
    To access and existing account, email and password need to be inserted and "Login" pressed. Use the "Remember me" function to automatically login upon the start of the application. 

    \section{Settings}
    The settings symbol allows the user to access the application settings. Currently, only changing the language of the application between English and German is possible. 
    
    \section{Vocabulary entry}
    Each vocabulary consists of
    \begin{itemize}
        \item Question
        \begin{itemize}
            \item The translation or description in the original language.
        \end{itemize}
        \item Answer
        \begin{itemize}
            \item The correct word in the target language, thus, the word to learn.
        \end{itemize}
        \item Language
        \begin{itemize}
            \item The language in which the word is learned.
        \end{itemize}
        \item Phase
        \begin{itemize}
            \item The phase describes the state of learning of each individual vocab: Newly added vocabulary start at the lowest phase and progress towards the maximum phase with correctly given answers.\\
            \textbf{Note: Differing from the project requirements, the lowest phase was set to 0 and the highest to 4, as this is defined by assigning phase 0 to new entries by the \href{https://api.voc5.org/}{API}!} \\
            Hence, new entries start at 0 and are assumed to be perfectly learned at phase 4.
        \end{itemize}
    \end{itemize}
    
    \section{Managing Vocabulary}
    The Vocabulary pane allows for adding new and editing existing vocables. Changes are automatically synchronized with the server and, hence, saved. \\
        
        \subsection{Editing and Deleting}
        It is possible to search for specific vocables by entering a search topic into the text field in the upper left corner and pressing the "enter" key or the "Search" button on the right hand side of the field. By pressing the "Reset" button the filters will be reset and the whole vocabulary will be shown again.
        To edit the vocabulary, enable editing by pressing the "Edit Mode" button:
        It is now possible to change the answer, the question, and the language of each entry in the table below by performing a double-click on the respective cell. Besides enabling editing, the previously hidden buttons "Add" and "Delete" are shown. Also, a column of check boxes  ("Select Column") will appear on the right side of the table. To delete selected vocables  press the "Delete" button after selecting the respective entries from the table.  
        Pressing the "Add" button opens the "Add" dialog.
  
        \subsection{Adding entries} 
        This Scene is responsible for adding new vocabulary. It consists of text fields for the answer, the question, and the language of the new vocable. By pressing the "Confirm" button in the lower left hand corner the input is converted, saved and added to the table. The entry will also be permanently stored on the server. The text fields are automatically cleared and can accept a new input. The "Cancel" button, located in the lower right hand corner, cancels all input and closes the "Add" dialog.
    
    \section{Learning}
        
        \subsection{Learning}
        In order to start learning, switch to the learning screen. If accessed directly, all of the user's vocabulary will be learned (see \cref{goals}). \\
        In the screen, a question fields from one of the user's vocables will appear together with a field for the answer. Enter the assumed translation and press solve, in order to show the right answer. The correctness of the user's answer is classified into 3 categories: Completely right, partially right (e.g. containing single typos), and wrong yielding a score of 3,1, and 0, respectively. A completely right answer will further increase the phase of the vocable by one, whereas a wrong answer will decrease it by one. A partially right answer will keep the vocable's phase unchanged. Phase changes will also be synchronized with the user's data stored on the server.\\
        If the question was assumed wrong or partially wrong (e.g. due to ambiguity or synonymous translation), the user can also correct the scoring by pressing "Manual correction" and selecting the right scoring category. \\
        After going through all vocables selected for learning, a overview of the learning results is shown.\\
        
        \subsection{Goals} \label{goals}
        In order to offer the option to only learn individual words or word in defined languages, the \textit{Goals} screen can be used. This pane supplies a list of all of the user's vocabulary at a phase lower than four. By pressing on "Filter Languages", the shown vocabulary can be subset to only specific languages. To ignore vocabulary from learning, the checkbox of each vocab can be unchecked, either for each entry manually, or by clicking on "Deselect all". The list of vocabulary can be sorted by any property by pressing on the respective column title. \\
        Upon pressing "Start Learning" and "Start Learning in Random Order" all selected vocabulary can be learned in the shown or random order, respectively.